% THIS IS THE FILE WITH THE PREAMBLE
\documentclass[12pt,openany,letterpaper,pagesize]{scrbook}

\usepackage[ansinew]{inputenc}
% \usepackage[spanish]{babel} &I am not actually writing in spanish
\usepackage[T1]{fontenc}
\usepackage{fancyhdr}
\usepackage{epsfig}
\usepackage{epic}
\usepackage{eepic}
\usepackage{amsmath}
\usepackage{breqn}
\usepackage{threeparttable}
\usepackage{amscd}
\usepackage{here}
\usepackage{graphicx}
\usepackage{lscape}
\usepackage{tabularx}
\usepackage[normalem]{ulem} % To slash terms
% \usepackage{subfigure}
\usepackage{subcaption} %Multi figures
\usepackage{epigraph}
\usepackage{dsfont} %The font of R of the reals
\usepackage{braket} %Dirac Notation
\usepackage{longtable}
\usepackage{tikz-cd} %The package that will change your life drawing commutation diagrams
\usepackage{tikz, pgfplots} %The package that will change your life
\usetikzlibrary{positioning}
\usepackage{hyperref} %Hyperlinks in the document
%Configure the bibliography
\usepackage[backend=biber, style=numeric,eprint=false, isbn=false, doi=false,
clearlang=true, url=false, sortcites]{biblatex}
%Prevent the note in the bib from printing
\DeclareSourcemap{
  \maps[datatype=bibtex]{
    \map{
      \step[fieldset=note, null]
    }
  }
}
%% Print the bibliography with the desired fields and order:
\addbibresource{Bibliography.bib}
% \nocite{*}
% \bibliographystyle{plaindin_esp}

\usepackage{rotating} %Para rotar texto, objetos y tablas seite. No se ve en DVI solo en PS. Seite 328 Hundebuch
                        %se usa junto con \rotate, \sidewidestable ....


%\renewcommand{\theequation}{\thechapter-\arabic{equation}}
%\renewcommand{\thefigure}{\textbf{\thechapter-\arabic{figure}}}
%\renewcommand{\thetable}{\textbf{\thechapter-\arabic{table}}}
%Define theorem environments
\newtheorem{definition}{Definition}[chapter]
\newtheorem{theorem}{Theorem}[chapter]
\newtheorem{lemma}{Lemma}[chapter]
\newtheorem{proof}{Proof}[chapter]
%%%%%%%%%%%%%%%%%%%%%%%%Useful for writing Dirac Notation more efficiently
\newcommand{\ketbra}[1]{\ket{#1}\hspace{-0.1cm}\bra{#1}}
\newcommand{\expectation}[2]{\bra{#1}\hspace{-0.05cm}#2\hspace{-0.05cm}\ket{#1}}
%%%%% Define the trace and the partiak trace
\newcommand{\Tr}[1]{\mathrm{Tr}\left[#1\right]}
\newcommand{\Trp}[2]{\mathrm{Tr}_{#1}\left[#2\right]}
%%%%%%%%%% Define imaginary and real parts
\renewcommand{\Im}[1]{\operatorname{Im}\left\{#1\right\}}
\renewcommand{\Re}[1]{\operatorname{Re}\left\{#1\right\}}
%%%%%%%%%%%%%%%% Define variance
\newcommand{\var}[1]{\text{Var}\left[#1\right]}
\pagestyle{fancyplain}%\addtolength{\headwidth}{\marginparwidth}
\textheight22.5cm \topmargin0cm \textwidth16.5cm
\oddsidemargin0.5cm \evensidemargin-0.5cm%
\renewcommand{\chaptermark}[1]{\markboth{\thechapter\; #1}{}}
\renewcommand{\sectionmark}[1]{\markright{\thesection\; #1}}
\lhead[\fancyplain{}{\thepage}]{\fancyplain{}{\rightmark}}
\rhead[\fancyplain{}{\leftmark}]{\fancyplain{}{\thepage}}
\fancyfoot{}
\thispagestyle{fancy}%


\addtolength{\headwidth}{0cm}
\unitlength1mm %Define la unidad LE para Figuras
% \mathindent0cm %Define la distancia de las formulas al texto,  fleqn las descentra
\marginparwidth0cm
\parindent0cm %Define la distancia de la primera linea de un parrafo a la margen

%Para tablas,  redefine el backschlash en tablas donde se define la posici\'{o}n del texto en las
%casillas (con \centering \raggedright o \raggedleft)
\newcommand{\PreserveBackslash}[1]{\let\temp=\\#1\let\\=\temp}
\let\PBS=\PreserveBackslash

%Espacio entre lineas
\renewcommand{\baselinestretch}{1.1}

%Neuer Befehl f\"{u}r die Tabelle Eigenschaften der Aktivkohlen
\newcommand{\arr}[1]{\raisebox{1.5ex}[0cm][0cm]{#1}}

%Neue Kommandos
\usepackage{Befehle}


%Trennungsliste
\hyphenation {Reaktor-ab-me-ssun-gen Gas-zu-sa-mmen-set-zung
Raum-gesch-win-dig-keit Durch-fluss Stick-stoff-gemisch
Ad-sorp-tions-tem-pe-ra-tur Klein-schmidt
Kohlen-stoff-Mole-kular-siebe Py-rolysat-aus-beu-te
Trans-port-vor-gan-ge}
