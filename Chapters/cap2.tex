\chapter{Markovian Open Systems}
% As commented in the previous chapter, in many situations we are interested in only a few degrees of freedoms that couple to many more
% e.g. an atom interacting with an electromagnetic field, in such a way that the description of the complete system is untracktable. For this
% reason it is of great importance to develop methods that allow for the construction of equations of motions for reduced density operators, which
% when exact give almost as intractable non-markovian dynamics like those described by the Nakajima-Zwanzig equation \cite{breuer2002theory} and
% that thus require approximations for their analytical or numerical study. In this chapter we review the
\section{Dynamical Semigroups}
The most accessible and studied type of open quantum systems are those which accept a description through a \textbf{markovian} and \textbf{time
  homogeneous} differential equation. Classical stochastic processes of this kind are characterized through the semigroup
formed by the family of conditional probabilities \footnote{Called a propagator in \cite{breuer2002theory}.}, which is parametrized by the
elapsed time, with this motivation one introduces the quantum analog called \textbf{Dynamical Semigroups}. The analogy is nuanced, as the
time evolution of any probability distribution associated with an observable will not satisfy the Kolmogorov consistency condition, and thus
no description as classical stochastic processes is avaliable \cite{strasberg2022quantum}, furthermore, the evolution of the state
will still be deterministic.

\begin{definition}
  A differentiable parametric family of quantum operations $\{\mathcal{E}_{\tau}\}_{\tau=0}^{\infty}$ such that
  \begin{align}
    \mathcal{E}_{\tau}(\mathcal{E}_{\tau'}(\rho)) =& \mathcal{E}_{\tau + \tau'}(\rho)\\
    \mathcal{E}_{0} = id
  \end{align}
  i.e. that has the semigroup property, is called a \textbf{Dynamical Semigroup}. Strictly, one also demands
  some additional technical conditions on the continouity to treat the case of infinite dimensional Hilbert Spaces \cite{lindblad1976generators}.
\end{definition}

In general these are irreversible; mathematically because the image is \textit{contractive} and physically due to its
positive entropy production \cite{breuer2002theory}, for this demanding a full group structure is too strong. The main result
of this chapter is the classification in terms of generators due to Linblad, which is presented in the following theorem.

\begin{theorem}
  Given a dynamical semigroup $\{\mathcal{E}_{\tau}\}_{\tau=0}^{\infty}$ there exists a time independent superoperator $\mathcal{L}$, called
  \textbf{the generator}, such that:

  \begin{equation}\label{eq:linblad}
  \partial_{t}\rho(t) = \mathcal{L}\rho(t) = -i[ H,\rho ] + \sum_{k}\gamma_{k}\left(L_{k}\rho(t) L_{k}^{\dagger} - \frac{1}{2}\{L_{k}^{\dagger}L_{k}, \rho(t)\}\right)
  \end{equation}
  with $L_{k}$ operators  and $H$ a self-adjoint one, futhermore, all equations of this form define a dynamical semigroup.
  Equation (\ref{eq:linblad}) is called \textbf{the Linblad Master Equation}, the traceless $L_{k}$'s the \textbf{Linblad operators}, and
  the $\gamma_{k}$ are positive constants with inverse time dimensions \cite{breuer2002theory, lindblad1976generators, wiseman_quantum_2010,hornberger2009introduction}.
\end{theorem}

\section{Trajectory Interpretation of the Evolution}
It is clear that at the very least (\ref{eq:linblad}) accepts a formal solution in terms of an exponential superator $\exp(t\mathcal{L})$, one
can use this to form a generalized Dyson series expansion that will gives us a Kraus representation of it by defining a sort
of interaction picture. Begin by defining an
arbitrary decomposition of the generator in terms of two terms $\mathcal{L}_{0}, S$:
\begin{equation}
  \mathcal{L} = \mathcal{L}_{0} + S
\end{equation}
and now introduce the auxiliary unnormalized state $\rho'$
\begin{equation}
\rho = e^{\mathcal{L}_{0}t}\rho'
\end{equation}
subsituting it in the first equality of the Linblad equation one obtains:
\begin{align}
  e^{\mathcal{L}_{0}t}\partial_{t}\rho' + \mathcal{L}_{0}\rho =& S\rho + \mathcal{L}_{0}\rho \\
\partial_{t}\rho' =& e^{-\mathcal{L}_{0}t}Se^{\mathcal{L}_{0}t}\rho_{1}
\end{align}
and integrating from 0 to $t$:
\begin{align}
\rho' = \rho(0) + \int_{0}^{t}dt_{1} e^{-\mathcal{L}_{0}t_{1}}Se^{\mathcal{L}_{0}t_{1}}\rho_{1}(t_{1}).
\end{align}
Iterating this equation:

\begin{align}
\rho' =& \rho(0) +  \int_{0}^{t}dt_{1} e^{-\mathcal{L}_{0}t_{1}}Se^{\mathcal{L}_{0}t_{1}}\left(\rho(0) + \int_{0}^{t_{1}}dt_{2} e^{-\mathcal{L}_{0}t_{2}}Se^{\mathcal{L}_{0}t_{2}}\rho'(t_{2}) \right)\\
  \rho' =& \rho(0) +  \int_{0}^{t}dt_{1} e^{-\mathcal{L}_{0}t_{1}}Se^{\mathcal{L}_{0}t_{1}}\rho(0) + \int_{0}^{t}dt_{1} \int_{0}^{t_{1}}dt_{2}e^{-\mathcal{L}_{0}t_{1}}Se^{\mathcal{L}_{0}(t_{1}-t_{2})}Se^{\mathcal{L}_{0}t_{2}}\rho'(t_{2})\\
  \rho' =& \rho(0) + \sum_{n=1}^{\infty}\int_{0}^{t}dt_{n}\int_{0}^{t_{n}}dt_{n-1}...\int_{0}^{t_{2}}dt_{1}e^{-t_{n}\mathcal{L}_{0}}Se^{(t_{n}-t_{n-1})\mathcal{L}_{0}}S...e^{(t_{2}-t_{1})\mathcal{L}_{0}}Se^{t_{1}\mathcal{L}_{0}}\rho(0).
\end{align}
Note that in the last line we inverted the order of the indexation to make it coincide with \cite{hornberger2009introduction}, so finally
we have for the original state:
\begin{equation}\label{eq:unraveling}
  \rho = e^{t\mathcal{L}_{0}}\rho(0) + \sum_{n=1}^{\infty}\int_{0}^{t}dt_{n}\int_{0}^{t_{n}}dt_{n-1}...\int_{0}^{t_{2}}dt_{1}e^{(t-t_{n})\mathcal{L}_{0}}Se^{(t_{n}-t_{n-1})\mathcal{L}_{0}}S...e^{(t_{2}-t_{1})\mathcal{L}_{0}}Se^{t_{1}\mathcal{L}_{0}}\rho(0).
\end{equation}
This has a formally identical structure to that of the Dyson series in a pertubative expansion, suggesting that one can interpret the evolution
of the system as being given by a perturbation $S$ to the evolution $e^{t{L}_{0}}$. The main problem with this is that in general the splitting
superoperators do not necessarily define a dynamical semigroup e.g. they could fail to map into a traceless operator, which is a consistency
condition. To obtain a more suitable interpretation we choose a particular decomposition, coming from the second equality of (\ref{eq:linblad})
\cite{hornberger2009introduction}:

\begin{align}
\mathcal{L}_{0} =& -i(\tilde{H}\rho - \rho\tilde{H}^{\dagger})\\
S =& \sum_{k}\mathcal{L}_{k}\\
  \tilde{H} =& H - \frac{i}{2}\sum_{k}\gamma_{k}L_{k}^{\dagger}L_{k}\\
  \mathcal{L}_{k} = & J[\sqrt{\gamma_{k}}L_{k}]\rho\
\end{align}
where we used the notation $J[A]\rho = A\rho A^{\dagger}$.
For the evaluation we introduce the following result:
\begin{lemma}
  For any superoperator of the form $\mathcal{A}\rho = A\rho+\rho A^{\dagger}$,
  \begin{equation}
    e^{\tau\mathcal{A}}\rho=\exp(\tau A)\rho\exp(\tau A^{\dagger})
  \end{equation}
\end{lemma}
  \begin{proof}
    Taking the derivative one forms the equation
    \begin{equation}
      \partial_{t}e^{\tau\mathcal{A}}\rho =  A\rho+\rho A^{\dagger}
    \end{equation}
    which by inspection one sees has the solution
    $e^{\tau\mathcal{A}}\rho=\exp(\tau A)\rho\exp(\tau A^{\dagger})$.
  \end{proof}
  allowing us to write (\ref{eq:unraveling}) as in \cite{wiseman_quantum_2010}:


  \begin{equation}\label{eq:superoperator_decomposition}
    \rho = J[e^{t \frac{1}{i}\tilde{H}}]\rho(0)  +  \sum_{n=1}^{\infty}\mathcal{K}^{(n)}_{t}\rho(0)
      \end{equation}
\begin{align}
  \mathcal{K}^{(n)}_{t}=&\int_{0}^{t}dt_{n}\int_{0}^{t_{n}}dt_{n-1}...\int_{0}^{t_{2}}dt_{1}J[e^{(t-t_{n}) \frac{1}{i}\tilde{H}}]SJ[e^{(t_{n}-t_{n-1}) \frac{1}{i}\tilde{H}}]...J[e^{(t_{2}-t_{1}) \frac{1}{i}\tilde{H}}]SJ[e^{t_{1}\frac{1}{i}\tilde{H}}]\\
  \mathcal{K}^{(n)}_{t}=&\sum_{k_{1}...k_{n}}\int_{0}^{t}dt_{n}\int_{0}^{t_{n}}dt_{n-1}...\int_{0}^{t_{2}}dt_{1}J\left[e^{(t-t_{n})\frac{1}{i}\tilde{H}}\sqrt{\gamma_{k_{n}}}L_{k_{n}}e^{(t_{n}-t_{n-1})\frac{1}{i}\tilde{H}}...e^{(t_{2}-t_{1})\frac{1}{i}\tilde{H}}\sqrt{\gamma_{k_{1}}}L_{k_{1}}e^{t_{1}\frac{1}{i}\tilde{H}}\right]
\end{align}
the integrands inside $\mathcal{K}_{t}^{(n)}$ are positive, and so they must be non-trace increasing for (\ref{eq:superoperator_decomposition})
to have the same trace in both sides. With this we conclude that (\ref{eq:unraveling}) can be interpreted as a piecewise deterministic process,
in which continous evolutions $\exp(\tau\mathcal{L}_{0})$ are interrupted by environment induced transformations $J[L_{k}]$ at a rate
$\gamma_{k}$. More precesely, the probability of the systems evolving during a time $t$ without \textbf{jumps} i.e. only continously is:

\begin{equation}
  P(R_{0}^{t}|\rho) = \Tr{e^{\mathcal{L}_{0}t}}
\end{equation}

and the probability of having $n$ jumps $k_{n},...,k_{1}$ at respective times $t>t_{n}>...>t_{1}$ is

\begin{equation}
  P(R_{n}^{t>t_{n}>...>t_{1}}|\rho) = \Tr{J\left[e^{(t-t_{n})\frac{1}{i}\tilde{H}}L_{k_{n}}e^{(t_{n}-t_{n-1})\frac{1}{i}\tilde{H}}...e^{(t_{2}-t_{1})\frac{1}{i}\tilde{H}}L_{k_{1}}e^{t_{1}\frac{1}{i}\tilde{H}}\right]\rho}.
\end{equation}

This is the interpretation of the \textbf{Quantum Trajectories} \cite{wiseman_quantum_2010,hornberger2009introduction}

\section{Damped Harmonic Osccilator}
To illustrate how Linblad equations appear in practice we look at a single mode of a QED cavity coupled to many others in the exterior of it,
we proceed with a \textbf{\textit{microscopic derivation}} in which we pose a total hamiltonian for system and bath and then introduce
approximations to obtain a Markovian completely positive evolution. We consider the following hamiltonian \cite{wiseman_quantum_2010,walls_quantum_2008}:

\begin{equation}\label{eq:radiative_hamiltonian}
H_{tot} = \underbrace{\overbrace{\omega_{c} a^{\dagger}a}^{H_{c}} + \overbrace{\sum_{k}\omega_{k}b^{\dagger}_{k}b_{k}}^{H_{E}} }_{H_{0}}+ \underbrace{\sum_{k}g_{k}\left( a^{\dagger}b_{k} + ab^{\dagger}_{k} \right)}_{H_{I}}.
\end{equation}
Where the $a$'s and $b_k$'s satisfy the bosonic commutation relation.
\subsection{Transformation to the Interaction Picture}
The interesting part of the dynamic lies in the coupling term of \eqref{eq:radiative_hamiltonian} and so we pass to an interaction picture; we
denote the Schrödinger picture operators with tildes.

\begin{align}
  \tilde{\rho}_{tot}(t) =& e^{-iH_{0}t} \rho_{tot}(t)e^{iH_{0}t}\\
  \partial_{t} \tilde{\rho}_{tot}(t) =& -i[H_{0}, \tilde{\rho}_{tot}(t)] - i[H_I,\tilde{\rho}_{tot}(t) ]\\
  -iH_{0}\tilde{\rho}_{tot} +i\tilde{\rho}_{tot}H_{0} + e^{-iH_{0}t}\partial_{t} \rho_{tot}(t)e^{iH_{0}t} =&-i[H_{0}, \tilde{\rho}_{tot}(t)] - i[H_I,\tilde{\rho}_{tot}(t) ]\\
\partial_{t} \rho_{tot}(t) =&-i[e^{iH_{0}t}H_Ie^{-iH_{0}t},\rho_{tot}(t) ]
\end{align}
 now obtaining the interaction hamiltonian:

\begin{align}
  e^{iH_{0}t} H_{I}e^{-iH_{0}t} =& \sum_{k}g_{k}\left(e^{iH_{c}t}a^{\dagger}e^{-iH_{c}t} e^{iH_{E}t}b_{k}e^{-iH_{E}t} +
                                   e^{iH_{c}t}ae^{-iH_{c}t} e^{iH_{E}t}b_{k}^{\dagger}e^{-iH_{E}t}\right)\\
  H\equiv & \sum_{k} g_{k} \left( e^{i(\omega_{c}-\omega_{k})t}a^{\dagger}b +   e^{-i(\omega_{c}-\omega_{k})t}ab^{\dagger} \right)
\end{align}
one is lead to the reduced Von-Neumann equation:
\begin{align}
  \partial_{t}\rho_{tot} =& -i[H, \rho_{tot}]\\
 \underbrace{\partial_{t}\rho}_{\Trp{E}{\partial_{t}\rho_{tot}}} =& -i\mathrm{Tr}_{E}\left\{[H(t), \rho_{tot}(0)]\right\} - \int_{0}^{t}dt'\mathrm{Tr}_{E}\left\{[H(t), [H(t'), \rho_{tot}(t')]]\right\}.\label{eq:pre_approximation}
\end{align}
So far the treatment has been exact and general but at this point the necessity for approximations come, these are typical for most derivations
of this type \cite{hornberger2009introduction,wiseman_quantum_2010,breuer2002theory}.
\subsubsection{Initial state of the total system}
Our first assumption will be that at $t=0$ system and bath are completely uncorrelated i.e. $\rho_{tot}(0)=\rho(0)\otimes \rho_{E}(0)$,
physically this is reasonable if enough control over the system to prepare its initial state in isolation from the environment.
Furthermore the first term of \eqref{eq:pre_approximation} is forced to vanish by choosing a particular initial state for the bath, in our case
we use a thermal state $\rho_{E}(0)=Z^{-1}e^{-\beta H_{E}}$\footnote{Here $Z$ is the canonical partition function and $\beta=1/kT$.}:

\begin{align}
  [H, \rho_{tot}(0)] =& Z^{-1}\left[ \sum_{k} g_{k} \left( e^{i(\omega_{c}-\omega_{k})t}a^{\dagger}b +   e^{-i(\omega_{c}-\omega_{k})t}ab^{\dagger}
                        \right),\rho(0)\otimes e^{-\beta H_{E}}\right]\\
   [H, \rho_{tot}(0)] =& Z^{-1}\sum_{k} g_{k}\left[  e^{i(\omega_{c}-\omega_{k})t}a^{\dagger}b +   e^{-i(\omega_{c}-\omega_{k})t}ab^{\dagger}
                        ,\rho(0)\otimes e^{-\beta H_{E}}\right]\\
  \begin{split}
  [H, \rho_{tot}(0)] =& Z^{-1} \sum_{k} g_{k}e^{i(\omega_{c}-\omega_{k})t}\left(a^{\dagger}\rho(0)\otimes b_{k}e^{-\beta H_{E}} -
                        \rho(0)a^{\dagger}\otimes e^{-\beta H_{E}}b_{k} \right)\\
                      & +Z^{-1} \sum_{k} g_{k}e^{-i(\omega_{c}-\omega_{k})t}\left(a\rho(0)\otimes b_{k}^{\dagger}e^{-\beta H_{E}} -
                        \rho(0)a\otimes e^{-\beta H_{E}}b_{k}^{\dagger} \right)
  \end{split}
\end{align}
now we marginalzie and evaluate the corresponding traces,
\begin{equation}
\begin{split}
  \Trp{E}{[H, \rho_{tot}(0)]} =& Z^{-1} \sum_{k} g_{k}e^{i(\omega_{c}-\omega_{k})t}\left(a^{\dagger}\rho(0)\Tr{b_{k}e^{-\beta H_{E}}} -
                        \rho(0)a^{\dagger}\Tr{e^{-\beta H_{E}}b_{k}} \right)\\
                      & +Z^{-1} \sum_{k} g_{k}e^{-i(\omega_{c}-\omega_{k})t}\left(a\rho(0)\Tr{b_{k}^{\dagger}e^{-\beta H_{E}}} -
                        \rho(0)a\Tr{e^{-\beta H_{E}}b_{k}^{\dagger}} \right)
  \end{split}
\end{equation}

\begin{equation}
  \Tr{b_{k}e^{-\beta H_{E}}}=\Tr{b_{k}e^{-\beta\omega_{k}b_{k}^{\dagger}b_{k}}\bigotimes_{k'\neq k}e^{-\beta\omega_{k'}b_{k'}^{\dagger}b_{k}}}= \underbrace{\Tr{b_{k}e^{-\beta\omega_{k}b_{k}^{\dagger}b_{k}}}}_{=0}\Tr{\bigotimes_{k'\neq k}e^{-\beta\omega_{k'}b_{k'}^{\dagger}b_{k}}}
\end{equation}
\begin{equation}
\Tr{b_{k}e^{-\beta H_{E}}}=\Tr{b_{k}^{\dagger}e^{-\beta H_{E}}}=0.
\end{equation}
Hence for any $\rho(0)$ we have:
\begin{equation}
\Trp{E}{[H, \rho_{tot}(0)]}=0
\end{equation}
and as a matter of fact, in general there always exists a choice of $\rho_{E}(0)$ that makes this term vanish \cite{wiseman_quantum_2010}.
\subsubsection{Born Approximation}
Restricting the values of the coupling paramerets $g_{k}$ to be small, we propose that in terms of second order like integral in the equation
above it is valid to make the approximation $\rho_{tot}(t)\approx \rho(t)\otimes\rho_{E}(0)$. This is essentially a \textbf{weak coupling}
assumption and is called the \textit{Born approximation}, which from the calculational point of view allows us to evaluate the double commutator in
\eqref{eq:pre_approximation}.

\begin{align}
  [H(t'), \rho(t')\otimes\rho_{E}(0)]=&\sum_{k}g_{k}e^{i(\omega_{c}-\omega_{k})t'}[a^{\dagger}b_{k},\rho(t')\otimes\rho_{E}(0)] +
g_{k}e^{-i(\omega_{c}-\omega_{k})t'}[ab_{k}^{\dagger},\rho(t')\otimes\rho_{E}(0)]\\
  \begin{split}
    [H(t'), \rho(t')\otimes\rho_{E}(0)]=&\sum_{k}g_{k}e^{i(\omega_{c}-\omega_{k})t'}\left\{a^{\dagger}\rho(t')\otimes b_{k}\rho_{E}(0) - \rho(t')a^{\dagger}\otimes\rho_{E}(0)b_{k} \right\}\\
+&\sum_{k}g_{k}e^{-i(\omega_{c}-\omega_{k})t'}\left\{a\rho(t')\otimes b_{k}^{\dagger}\rho_{E}(0) - \rho(t')a\otimes\rho_{E}(0)b_{k}a^{\dagger} \right\}
  \end{split}
\end{align}
now to evaluate the double commutator we use (NOTE: THIS GOES IN AN APPENDIX, THE CALCULATIONS ARE QUITE TEDIOUS AND LONG):
\begin{align}
  \Trp{E}{[H(t),a^{\dagger}\rho(t')\otimes b_{k}\rho_{E}(0)]} =& g_{k}e^{-i(\omega_{c}-\omega_{k})t}\Tr{b_{k}^{\dagger}b_{k}\rho_{E}(0)}(aa^{\dagger}\rho(t')-a^{\dagger}\rho(t')a)\\
  \Trp{E}{[H(t),\rho(t')a^{\dagger}\otimes \rho_{E}(0)b_{k}]} =& g_{k}e^{-i(\omega_{c}-\omega_{k})t}\Tr{b_{k}b_{k}^{\dagger}\rho_{E}(0)}(a\rho(t')a^{\dagger} - \rho(t')a^{\dagger}a)\\
\Trp{E}{[H(t),a\rho(t')\otimes b_{k}^{\dagger}\rho_{E}(0)]} =& g_{k} e^{i(\omega_{c}-\omega_{k})t}\underbrace{\Tr{b_{k}b_{k}^{\dagger}\rho_{E}(0)}}_{\overline{n}+1}(a^{\dagger}a\rho(t') - a\rho(t')a^{\dagger})\\
\Trp{E}{[H(t),\rho(t')a\otimes \rho_{E}(0)b_{k}^{\dagger}]} =& g_{k}e^{i(\omega_{c}-\omega_{k})t}\underbrace{\Tr{b_{k}^{\dagger}b_{k}\rho_{E}(0)}}_{\overline{n}}(a^{\dagger}\rho(t')a - \rho(t')aa^{\dagger})
\end{align}
where $\overline{n}$ is the average number of photons in the bath at the corresponding temperature.
\begin{equation}
\begin{split}
  [H(t), [H(t'), \rho(t')\otimes \rho_{E}(0)]] =& \sum_{k}g_{k}^{2}e^{i(\omega_{c}-\omega_{k})(t'-t)}\left\{\overline{n}(aa^{\dagger}\rho(t')-a^{\dagger}\rho(t')a) - (\overline{n}+1)(a\rho(t')a^{\dagger} - \rho(t')a^{\dagger}a) \right\}\\
  &\hspace{-1cm}+\sum_{k}g_{k}^{2}e^{-i(\omega_{c}-\omega_{k})(t'-t)}\left\{(\overline{n}+1)(a^{\dagger}a\rho(t')-a\rho(t')a^{\dagger}) - (\overline{n})(a^{\dagger}\rho(t')a - \rho(t')aa^{\dagger}) \right\}
\end{split}
\end{equation}
Now we define the \textit{\textbf{Bath correlation function}}\footnote{called Reservoire correlation function in \cite{wiseman_quantum_2010}} as:

\begin{equation}
  \Gamma(\tau) =\sum_{k}g_{k}^{2}e^{i(\omega_{c}-\omega_{k})\tau}\label{eq:bath_correlation}
\end{equation}

one gets

\begin{equation}
  \begin{split}
    [H(t), [H(t'), \rho(t')\otimes \rho_{E}(0)]] =& -2(\overline{n}+1)\Re{(\Gamma(t-t'))}a\rho(t')a^{\dagger} -2 \overline{n}\Re{\Gamma(t-t')}a^{\dagger}\rho(t')a\\
                                                  &+\Re{\Gamma{t-t'}}(\overline(n)+1)\{a^{\dagger}a, \rho{t'}\} - i\Im{\Gamma(t-t')}(\overline(n)+1)[a^{\dagger}a,\rho(t')]\\
    &+\Re{\Gamma(t-t')}\overline{n}\{aa^{\dagger}, \rho(t')\} -i \Im{\Gamma(t-t')}\overline{n}[aa^{\dagger},\rho(t')]
  \end{split}
\end{equation}
substituting this into \eqref{eq:pre_approximation} gives a non-markovian equation called a \textit{Redfield equation}.
\subsubsection{Markov Approximation}
Finally, to make the equation markovian we demand the \eqref{eq:bath_correlation} be very sharp at $\tau=0$ so that the integral in
\eqref{eq:pre_approximation} can have its lower limit extended to $-\infty$ and $\rho(t')$ be replaced by $\rho(t)$. Physically this approximatio
n corresponds to the bath having a correlation function that decays very rapidly \cite{hornberger2009introduction}, for which we define:

\begin{equation}
  \int_{0}^{\infty}\Gamma(\tau)=\frac{\gamma}{2} -i \Delta \omega_{c}
\end{equation}
and finally obtain a markovian master equation, which is already in Linblad form with the supeoperators $\mathcal{D}[A]\rho=A\rho A^{\dagger}-\frac{1}{2}\{A^{\dagger}A,\rho\}$\footnote{these are called \textit{Dissipators}}:

\begin{equation}
  \partial_{t}\rho = \underbrace{-i\Delta\omega_{c}[(\overline{n}+1)a^{\dagger}a + \overline{n}aa^{\dagger}, \rho]}_{\text{coherent evolution}} + \underbrace{(\overline{n}+1)\gamma\mathcal{D}[a]\rho +
\overline{n}\gamma\mathcal{D}[a^{\dagger}]\rho}_{\text{incoherent evolution}}.
\end{equation}
By rewriting the first argument of the commutator  in the first term, using the bosonic commutation relation, we see that the presence of the bath modifies natural frequency of the system $\Delta\omega_{c}$ in a fashion similar to how atoms suffer frequency shifts when placed into cavities even at zero temperature \cite{dutra2005cavity}

\begin{equation}
\partial_{t}\rho = \underbrace{-i\Delta\omega_{c}(\overline{n}+2)[aa^{\dagger}, \rho]}_{\text{photon number preserving}} + \underbrace{(\overline{n}+1)\gamma\mathcal{D}[a]\rho}_{\text{cavity emission}} +\underbrace{\overline{n}\gamma\mathcal{D}[a^{\dagger}]\rho}_{\text{incoherent excitation}}.
\end{equation}
To understand the other two we form a equation for the mean photon number of the mode by multiplying the above equation by $a^{\dagger}$ and
taking the trace:

\begin{align}
  \Tr{a^{\dagger}a\mathcal{D}[a]\rho} =& \Tr{a^{\dagger}aa\rho a^{\dagger}}-\frac{1}{2}\Tr{a^{\dagger}aa^{\dagger}a\rho}-\frac{1}{2}\Tr{a^{\dagger}a\rho a^{\dagger}a} \\
  \Tr{a^{\dagger}a\mathcal{D}[a]\rho} =& \Tr{(aa^{\dagger}-1)a\rho a^{\dagger}} - \Tr{a^{\dagger}aa^{\dagger}a\rho}\\
  \Tr{a^{\dagger}a\mathcal{D}[a]\rho} =&-\braket{a^{\dagger}a}\\
%\end{align}
% \begin{align}
\Tr{a^{\dagger}a\mathcal{D}[a^{\dagger}]\rho} =& \Tr{a^{\dagger}aa^{\dagger}\rho a} - \frac{1}{2}\Tr{a^{\dagger}aaa^{\dagger}\rho} - \frac{1}{2}\Tr{a^{\dagger}a\rho aa^{\dagger}}\\
\Tr{a^{\dagger}a\mathcal{D}[a^{\dagger}]\rho} =& \Tr{a^{\dagger}aa^{\dagger}\rho a} - \frac{1}{2}\Tr{(aa^{\dagger}-1)aa^{\dagger}\rho} - \frac{1}{2}\Tr{(aa^{\dagger}-1)\rho aa^{\dagger}}\\
\Tr{a^{\dagger}a\mathcal{D}[a^{\dagger}]\rho} =& 1+ \braket{a^{\dagger}a}
\end{align}
From here it is clear that the second term refers to the emission of the cavity, and the third one to the absortion from the bath. Substituting
we find:
\begin{align}
  \frac{d}{dt}\braket{a^{\dagger}a} = &-\gamma\braket{a^{\dagger}a} + \overline{n}\gamma\\
  \braket{a^{\dagger}a}(t) =& (\braket{a^{\dagger}a}(0)-\overline{n})e^{-\gamma t} + \overline{n},
\end{align}
and so the mode tends to equilibrate in photon number with the bath when $\gamma t\gg 1$.
%%% Local Variables:
%%% mode: latex
%%% TeX-master: "../main"
%%% End:
