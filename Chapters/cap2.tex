\chapter{Markovian Open Systems}
As commented in the previous chapter, in many situations we are interested in only a few degrees of freedoms that couple to many more
e.g. an atom interacting with an electromagnetic field, in such a way that the description of the complete system is untracktable. For this
reason it is of great importance to construct equations of motions for reduced density operators, nevertheless
when exact they tend to give almost as intractable problems; like solving the Nakajima-Zwanzig equation \cite{breuer2002theory}, for this it is
necessary to develop a approximation schemes that allow the study of certain regimes, among these by far the most studied one is the markovian
regime which we describe and apply in the context of quantum optics during this chapter.
\\\\
We begin by characterizing markovian open systems as those possesing a evolution described by a dynamical semigroup, immedeatly after
we state Lindblad's theorem to describe the equations of motion that these obey. Then we introduce the idea of quantum trajectories
to interpret the evolution. Finally study a damped QED cavity to show the typical approximations that produce a markovian equation
of motion, this will also give us our first example of a bosonic channel.

\section{Dynamical Semigroups}
The most accessible and studied type of open quantum systems are those which accept a description through a \textbf{markovian} and \textbf{time
  homogeneous} differential equation. Classical stochastic processes of this kind are characterized through the semigroup
formed by the family of conditional probabilities \footnote{Called a propagator in \cite{breuer2002theory}.}, which is parametrized by the
elapsed time, with this motivation one introduces the quantum analog called \textbf{Dynamical Semigroups}. The analogy is nuanced, as the
time evolution of any probability distribution associated with an observable will not satisfy the Kolmogorov consistency condition, and thus
no description as classical stochastic processes is avaliable \cite{strasberg2022quantum}, furthermore, the evolution of the state
will still be deterministic.

\begin{definition}
  A differentiable parametric family of quantum operations $\{\mathcal{E}_{\tau}\}_{\tau=0}^{\infty}$ such that
  \begin{align}
    \mathcal{E}_{\tau}(\mathcal{E}_{\tau'}(\rho)) =& \mathcal{E}_{\tau + \tau'}(\rho)\\
    \mathcal{E}_{0} = id
  \end{align}
  i.e. that has the semigroup property, is called a \textbf{Dynamical Semigroup}. Strictly, one also demands
  some additional technical conditions on the continouity to treat the case of infinite dimensional Hilbert Spaces \cite{lindblad1976generators}.
\end{definition}

In general these are irreversible; mathematically because the image is \textit{contractive} and physically due to its
positive entropy production \cite{breuer2002theory}, and so demanding a full group structure is too strong. The main result
of this chapter is the classification in terms of generators due to Linblad, which is presented in the following theorem.

\begin{theorem}
  Given a dynamical semigroup $\{\mathcal{E}_{\tau}\}_{\tau=0}^{\infty}$ there exists a time independent superoperator $\mathcal{L}$, called
  \textbf{the generator}, such that:

  \begin{equation}\label{eq:linblad}
  \partial_{t}\rho(t) = \mathcal{L}\rho(t) = -i[ H,\rho ] + \sum_{k}\gamma_{k}\left(L_{k}\rho(t) L_{k}^{\dagger} - \frac{1}{2}\{L_{k}^{\dagger}L_{k}, \rho(t)\}\right)
  \end{equation}
  with $L_{k}$ operators  and $H$ a self-adjoint one, futhermore, all equations of this form define a dynamical semigroup.
  Equation (\ref{eq:linblad}) is called \textbf{the Gorini--Kossakowski-–Sudarshan--Linblad  Master Equation}\footnote{From here on we simply refer to it as Lindblad equation}, the traceless $L_{k}$'s the \textbf{Linblad operators}, and
  the $\gamma_{k}$ are positive constants with inverse time dimensions \cite{breuer2002theory, lindblad1976generators, wiseman_quantum_2010,hornberger2009introduction}.
\end{theorem}

\section{Trajectory Interpretation of the Evolution}
It is clear that at the very least (\ref{eq:linblad}) accepts a formal solution in terms of an exponential superator $\exp(t\mathcal{L})$,
this can be used to form a generalized Dyson series expansion that will gives us a Kraus representation of it by defining a sort
of interaction picture. Begin by defining an
arbitrary decomposition of the generator in terms of two terms $\mathcal{L}_{0}, S$
\begin{equation}
  \mathcal{L} = \mathcal{L}_{0} + S
\end{equation}
and now introduce the auxiliary unnormalized state $\rho'$,
\begin{equation}
\rho = e^{\mathcal{L}_{0}t}\rho'
\end{equation}
replaving it in the first equality of the Linblad equation, one obtains:
\begin{align}
  e^{\mathcal{L}_{0}t}\partial_{t}\rho' + \mathcal{L}_{0}\rho =& S\rho + \mathcal{L}_{0}\rho \\
\partial_{t}\rho' =& e^{-\mathcal{L}_{0}t}Se^{\mathcal{L}_{0}t}\rho,
\end{align}
and integrating from 0 to $t_{1}$:
\begin{align}
\rho' = \rho'(0) + \int_{0}^{t}dt_{1} e^{-\mathcal{L}_{0}t_{1}}Se^{\mathcal{L}_{0}t_{1}}\rho(t_{1}).
\end{align}
Iterating this relation:

\begin{equation}
\begin{split}
\rho' =& \rho(0) +  \int_{0}^{t}dt_{1} e^{-\mathcal{L}_{0}t_{1}}Se^{\mathcal{L}_{0}t_{1}}\left(\rho(0) + \int_{0}^{t_{1}}dt_{2} e^{-\mathcal{L}_{0}t_{2}}Se^{\mathcal{L}_{0}t_{2}}\rho'(t_{2}) \right)\\
  \rho' =& \rho(0) +  \int_{0}^{t}dt_{1} e^{-\mathcal{L}_{0}t_{1}}Se^{\mathcal{L}_{0}t_{1}}\rho(0) + \int_{0}^{t}dt_{1} \int_{0}^{t_{1}}dt_{2}e^{-\mathcal{L}_{0}t_{1}}Se^{\mathcal{L}_{0}(t_{1}-t_{2})}Se^{\mathcal{L}_{0}t_{2}}\rho'(t_{2})\\
  \rho' =& \rho(0) + \sum_{n=1}^{\infty}\int_{0}^{t}dt_{n}\int_{0}^{t_{n}}dt_{n-1}...\int_{0}^{t_{2}}dt_{1}e^{-t_{n}\mathcal{L}_{0}}Se^{(t_{n}-t_{n-1})\mathcal{L}_{0}}S...e^{(t_{2}-t_{1})\mathcal{L}_{0}}Se^{t_{1}\mathcal{L}_{0}}\rho(0).
\end{split}
\end{equation}
Note that in the last line we inverted the order of the indexation to make it coincide with \cite{hornberger2009introduction}, so finally
we have for the original state:
\begin{equation}\label{eq:unraveling}
  \rho = e^{t\mathcal{L}_{0}}\rho(0) + \sum_{n=1}^{\infty}\int_{0}^{t}dt_{n}\int_{0}^{t_{n}}dt_{n-1}...\int_{0}^{t_{2}}dt_{1}e^{(t-t_{n})\mathcal{L}_{0}}Se^{(t_{n}-t_{n-1})\mathcal{L}_{0}}S...e^{(t_{2}-t_{1})\mathcal{L}_{0}}Se^{t_{1}\mathcal{L}_{0}}\rho(0).
\end{equation}
This has a formally identical structure to that of the Dyson series in a pertubative expansion, suggesting that one can interpret the evolution
of the system as being given by a perturbation $S$ to the evolution $e^{t{L}_{0}}$. The main problem with this, is that in general the splitting
superoperators do not necessarily define a dynamical semigroup e.g. they could fail to map into a traceless operator, which is a consistency
condition. To obtain a more suitable interpretation we choose a particular decomposition, coming from the second equality of (\ref{eq:linblad})
\cite{hornberger2009introduction}:

\begin{align}
\mathcal{L}_{0} =& -i(\tilde{H}\rho - \rho\tilde{H}^{\dagger}),\\
S =& \sum_{k}\mathcal{L}_{k},\\
  \tilde{H} =& H - \frac{i}{2}\sum_{k}\gamma_{k}L_{k}^{\dagger}L_{k},\\
  \mathcal{L}_{k} = & J[\sqrt{\gamma_{k}}L_{k}]\rho\,
\end{align}
where we used the notation $J[A]\rho = A\rho A^{\dagger}$.
For the evaluation we introduce the following result \footnote{This follows from taking the derivative of the exponential and inspecting for solutions}:
\begin{lemma}
  For any superoperator of the form $\mathcal{A}\rho = A\rho+\rho A^{\dagger}$,
  \begin{equation}
    e^{\tau\mathcal{A}}\rho=\exp(\tau A)\rho\exp(\tau A^{\dagger}).
  \end{equation}
\end{lemma}

Now we can write (\ref{eq:unraveling}) as:

  \begin{equation}\label{eq:superoperator_decomposition}
    \rho = J[e^{t \frac{1}{i}\tilde{H}}]\rho(0)  +  \sum_{n=1}^{\infty}\mathcal{K}^{(n)}_{t}\rho(0)
      \end{equation}
\begin{equation}
  \begin{split}
  \mathcal{K}^{(n)}_{t}=&\int_{0}^{t}dt_{n}\int_{0}^{t_{n}}dt_{n-1}...\int_{0}^{t_{2}}dt_{1}J[e^{(t-t_{n}) \frac{1}{i}\tilde{H}}]SJ[e^{(t_{n}-t_{n-1}) \frac{1}{i}\tilde{H}}]...J[e^{(t_{2}-t_{1}) \frac{1}{i}\tilde{H}}]SJ[e^{t_{1}\frac{1}{i}\tilde{H}}]\\
  \mathcal{K}^{(n)}_{t}=&\sum_{k_{1}...k_{n}}\int_{0}^{t}dt_{n}\int_{0}^{t_{n}}dt_{n-1}...\int_{0}^{t_{2}}dt_{1}J\left[e^{(t-t_{n})\frac{1}{i}\tilde{H}}\sqrt{\gamma_{k_{n}}}L_{k_{n}}e^{(t_{n}-t_{n-1})\frac{1}{i}\tilde{H}}...e^{(t_{2}-t_{1})\frac{1}{i}\tilde{H}}\sqrt{\gamma_{k_{1}}}L_{k_{1}}e^{t_{1}\frac{1}{i}\tilde{H}}\right].
  \end{split}
\end{equation}
The integrands inside $\mathcal{K}_{t}^{(n)}$ are positive, and so they must be non-trace increasing for (\ref{eq:superoperator_decomposition})
to have the same trace in both sides. With this we conclude that (\ref{eq:unraveling}) can be interpreted as a piecewise deterministic process,
in which continous evolutions $\exp(\tau\mathcal{L}_{0})$ are interrupted by environment induced transformations $J[L_{k}]$ at a rate
$\gamma_{k}$. More precesely, the probability of the systems evolving during a time $t$ without \textbf{jumps} i.e. only continuously is

\begin{equation}
  P(R_{0}^{t}|\rho) = \Tr{e^{\mathcal{L}_{0}t}}
\end{equation}

and the probability of having $n$ jumps $k_{n},...,k_{1}$ at respective times $t>t_{n}>...>t_{1}$ is

\begin{equation}
  P(R_{n}^{t>t_{n}>...>t_{1}}|\rho) = \Tr{J\left[e^{(t-t_{n})\frac{1}{i}\tilde{H}}L_{k_{n}}e^{(t_{n}-t_{n-1})\frac{1}{i}\tilde{H}}...e^{(t_{2}-t_{1})\frac{1}{i}\tilde{H}}L_{k_{1}}e^{t_{1}\frac{1}{i}\tilde{H}}\right]\rho}.
\end{equation}

A particular realization of jumps and continuous evolution is called a \textbf{Quantum Trajectory} \cite{wiseman_quantum_2010,hornberger2009introduction}.

\section{Damped Harmonic Osccilator}
To illustrate how Linblad equations appear in practice we look at a single mode of a QED cavity coupled to many others in the exterior of it,
proceeding with a \textbf{\textit{microscopic derivation}} in which a total hamiltonian for system and bath is posed,  and then introduce
approximations to obtain a Markovian completely positive evolution. We consider the following hamiltonian
\cite{wiseman_quantum_2010,walls_quantum_2008}:

\begin{equation}\label{eq:radiative_hamiltonian}
H_{tot} = \underbrace{\overbrace{\omega_{c} a^{\dagger}a}^{H_{c}} + \overbrace{\sum_{k}\omega_{k}b^{\dagger}_{k}b_{k}}^{H_{E}} }_{H_{0}}+ \underbrace{\sum_{k}g_{k}\left( a^{\dagger}b_{k} + ab^{\dagger}_{k} \right)}_{H_{I}}.
\end{equation}
Where the $a$'s and $b_k$'s satisfy the bosonic commutation relation.
\subsection{Transformation to the Interaction Picture}
The interesting part of the dynamic lies in the coupling term of \eqref{eq:radiative_hamiltonian}, and so we pass to an interaction picture. The
Schr{\"o}dinger picture operators are denoted with tildes.

\begin{align}
  \tilde{\rho}_{tot}(t) =& e^{-iH_{0}t} \rho_{tot}(t)e^{iH_{0}t}\\
  \partial_{t} \tilde{\rho}_{tot}(t) =& -i[H_{0}, \tilde{\rho}_{tot}(t)] - i[H_I,\tilde{\rho}_{tot}(t) ]\\
  -iH_{0}\tilde{\rho}_{tot} +i\tilde{\rho}_{tot}H_{0} + e^{-iH_{0}t}\partial_{t} \rho_{tot}(t)e^{iH_{0}t} =&-i[H_{0}, \tilde{\rho}_{tot}(t)] - i[H_I,\tilde{\rho}_{tot}(t) ]\\
\partial_{t} \rho_{tot}(t) =&-i[e^{iH_{0}t}H_Ie^{-iH_{0}t},\rho_{tot}(t) ]
\end{align}
\begin{align}
  e^{iH_{0}t} H_{I}e^{-iH_{0}t} =& \sum_{k}g_{k}\left(e^{iH_{c}t}a^{\dagger}e^{-iH_{c}t} e^{iH_{E}t}b_{k}e^{-iH_{E}t} +
                                   e^{iH_{c}t}ae^{-iH_{c}t} e^{iH_{E}t}b_{k}^{\dagger}e^{-iH_{E}t}\right)\\
  H\equiv & \sum_{k} g_{k} \left( e^{i(\omega_{c}-\omega_{k})t}a^{\dagger}b +   e^{-i(\omega_{c}-\omega_{k})t}ab^{\dagger} \right).
\end{align}
And so the equation of motion for the reduced density operator is obtained:
\begin{align}
  \partial_{t}\rho_{tot} =& -i[H, \rho_{tot}]\\
 \underbrace{\partial_{t}\rho}_{\Trp{E}{\partial_{t}\rho_{tot}}} =& -i\mathrm{Tr}_{E}\left\{[H(t), \rho_{tot}(0)]\right\} - \int_{0}^{t}dt'\mathrm{Tr}_{E}\left\{[H(t), [H(t'), \rho_{tot}(t')]]\right\}\label{eq:pre_approximation}
\end{align}
So far the treatment has been exact and general but at this point the necessity for approximations arises, here the most typical ones for this
type of derivation are used \cite{hornberger2009introduction,wiseman_quantum_2010,breuer2002theory}.
\subsubsection{Initial state of the total system}
Our first assumption will be that at $t=0$ system and bath are completely uncorrelated i.e. $\rho_{tot}(0)=\rho(0)\otimes \rho_{E}(0)$,
physically this is reasonable if enough control over the system to prepare its initial state in isolation from the environment.
Furthermore the first term of \eqref{eq:pre_approximation} is forced to vanish by choosing a particular initial state for the bath, in our case
we use a thermal state $\rho_{E}(0)=Z^{-1}e^{-\beta H_{E}}$\footnote{Here $Z$ is the canonical partition function and $\beta=1/kT$.}. As a matter of fact, in general there always exists a choice of $\rho_{E}(0)$ that makes this term vanish \cite{wiseman_quantum_2010}. See appendix (\ref{appendixA}) for a detailed calculation showing this is the case.

\subsubsection{Born Approximation}
Restricting the values of the coupling paramerets $g_{k}$ to be small, we propose that in the integral it is valid to make the approximation $\rho_{tot}(t)\approx \rho(t)\otimes\rho_{E}(0)$. This is essentially a \textbf{weak coupling}
assumption and is called the \textit{Born approximation}, which from the calculational point of view allows us to evaluate the double commutator in
\eqref{eq:pre_approximation}. In appendix (\ref{appendixB}) the calculation is perfomed and it is found that:

\begin{equation}
  \begin{split}
    [H(t), [H(t'), \rho(t')\otimes \rho_{E}(0)]] =& -2(\overline{n}+1)\Re{(\Gamma(t-t'))}a\rho(t')a^{\dagger} -2 \overline{n}\Re{\Gamma(t-t')}a^{\dagger}\rho(t')a\\
                                                  &+\Re{\Gamma{t-t'}}(\overline(n)+1)\{a^{\dagger}a, \rho{t'}\} - i\Im{\Gamma(t-t')}(\overline(n)+1)[a^{\dagger}a,\rho(t')]\\
    &+\Re{\Gamma(t-t')}\overline{n}\{aa^{\dagger}, \rho(t')\} -i \Im{\Gamma(t-t')}\overline{n}[aa^{\dagger},\rho(t')]
  \end{split}
\end{equation}

where  $\Gamma$, called the \textbf{Bath correlation function}, is
given by:

\begin{equation}
  \Gamma(\tau) =\sum_{k}g_{k}^{2}e^{i(\omega_{c}-\omega_{k})\tau}\label{eq:bath_correlation}.
\end{equation}

\subsubsection{Markov Approximation}
Finally, to make the equation markovian we demand the bath correlation be very sharp at $\tau=0$ so that the integral in
\eqref{eq:pre_approximation} can have its lower limit extended to $-\infty$ and $\rho(t')$ be replaced by $\rho(t)$. Physically this
approximation corresponds to the bath having a correlation function that decays very rapidly \cite{hornberger2009introduction}. We define:

\begin{equation}
  \int_{0}^{\infty}\Gamma(\tau)=\frac{\gamma}{2} -i \Delta \omega_{c}
\end{equation}
and obtain a markovian master equation, which is already in Linblad form with the supeoperators $\mathcal{D}[A]\rho=A\rho A^{\dagger}-\frac{1}{2}\{A^{\dagger}A,\rho\}$\footnote{these are called \textit{Dissipators}}:

\begin{equation}
  \partial_{t}\rho = \underbrace{-i\Delta\omega_{c}[(\overline{n}+1)a^{\dagger}a + \overline{n}aa^{\dagger}, \rho]}_{\text{coherent evolution}} + \underbrace{(\overline{n}+1)\gamma\mathcal{D}[a]\rho +
\overline{n}\gamma\mathcal{D}[a^{\dagger}]\rho}_{\text{incoherent evolution}}.
\end{equation}
By rewriting the first argument of the commutator  in the first term, using the bosonic commutation relation, we see that the presence of the bath modifies natural frequency of the system $\Delta\omega_{c}$ in a fashion similar to how atoms suffer frequency shifts when placed into cavities even at zero temperature \cite{dutra2005cavity}

\begin{equation}
\partial_{t}\rho = \underbrace{-i\Delta\omega_{c}(\overline{n}+2)[aa^{\dagger}, \rho]}_{\text{photon number preserving}} + \underbrace{(2\overline{n}+1)\gamma\mathcal{D}[a]\rho}_{\text{cavity emission}} +\underbrace{\overline{n}\gamma\mathcal{D}[a^{\dagger}]\rho}_{\text{incoherent excitation}}.\label{eq:cavity_model}
\end{equation}
To understand the other two we form a equation for the mean photon number of the mode by multiplying the above equation by $a^{\dagger}$ and
taking the trace:

\begin{align}
  \Tr{a^{\dagger}a\mathcal{D}[a]\rho} =& \Tr{a^{\dagger}aa\rho a^{\dagger}}-\frac{1}{2}\Tr{a^{\dagger}aa^{\dagger}a\rho}-\frac{1}{2}\Tr{a^{\dagger}a\rho a^{\dagger}a} \\
  \Tr{a^{\dagger}a\mathcal{D}[a]\rho} =& \Tr{(aa^{\dagger}-1)a\rho a^{\dagger}} - \Tr{a^{\dagger}aa^{\dagger}a\rho}\\
  \Tr{a^{\dagger}a\mathcal{D}[a]\rho} =&-\braket{a^{\dagger}a}\\
%\end{align}
% \begin{align}
\Tr{a^{\dagger}a\mathcal{D}[a^{\dagger}]\rho} =& \Tr{a^{\dagger}aa^{\dagger}\rho a} - \frac{1}{2}\Tr{a^{\dagger}aaa^{\dagger}\rho} - \frac{1}{2}\Tr{a^{\dagger}a\rho aa^{\dagger}}\\
\Tr{a^{\dagger}a\mathcal{D}[a^{\dagger}]\rho} =& \Tr{a^{\dagger}aa^{\dagger}\rho a} - \frac{1}{2}\Tr{(aa^{\dagger}-1)aa^{\dagger}\rho} - \frac{1}{2}\Tr{(aa^{\dagger}-1)\rho aa^{\dagger}}\\
\Tr{a^{\dagger}a\mathcal{D}[a^{\dagger}]\rho} =& 1+ \braket{a^{\dagger}a}
\end{align}
From here it is clear that the second term refers to the emission of the cavity, and the third one to the absortion from the bath. Substituting
we find:
\begin{align}
  \frac{d}{dt}\braket{a^{\dagger}a} = &-\gamma\braket{a^{\dagger}a} + \overline{n}\gamma\\
  \braket{a^{\dagger}a}(t) =& (\braket{a^{\dagger}a}(0)-\overline{n})e^{-\gamma t} + \overline{n},
\end{align}
and so the mode tends to equilibrate in photon number with the bath when $\gamma t\gg 1$.
\subsection{Solution to the Cavity Emission Model}
A solution to \eqref{eq:cavity_model} can be constructed by exploting the bosonic commutation relations
to evaluate the superoperators $\mathcal{K}^{(n)}_{t}$. To simplify the treatment we remove the coherent term by passing to a new rotating frame
with the frequency shift times $2\overline{n}+1$, so the equation is reduced to \footnote{note that the dissipators are left unaltered thanks to the commutation relation.}:

\begin{equation}
\partial_{t}\rho = \gamma_{1}\mathcal{D}[a]\rho + \gamma_{2}\mathcal{D}[a^{\dagger}]\rho
\end{equation}

with $\gamma_{2} = \overline{n}\gamma$, $\gamma_{1} = (\overline{n}+1)\gamma$. From\eqref{eq:unraveling} we have

\begin{equation}\label{eq:kn_operators}
  \mathcal{K}_{t}^{(m)}=e^{\mathcal{L}_{0}t}\int_{0}^{t}dt_{m}e^{-t_{m}\mathcal{L}_{0}}Se^{t_{m}\mathcal{L}_{0}}\int_{0}^{t_{m}}dt_{m-1}e^{-t_{m-1}\mathcal{L}_{0}}Se^{t_{m-1}\mathcal{L}_{0}}...\int_{0}^{t_{2}}dt_{1}e^{-t_{1}\mathcal{L}_{0}}Se^{t_{1}\mathcal{L}_{0}},
\end{equation}

so it will suffice to evaluate the integrals and express them in terms of the canonical operators.

\begin{equation}
  \tilde{H} = -i\gamma\left((\overline{n}+1)a^{\dagger}a + \overline{n}aa^{\dagger} \right) = -i\gamma\left((2\overline{n}+1)a^{\dagger}a +\overline{n}  \right)
\end{equation}

\begin{equation}
  S = \gamma_{1}J[a]+\gamma_{2}J[a^{\dagger}]
\end{equation}

\begin{equation}
  e^{\mathcal{L}_{0}\tau} = e^{-\gamma\tau\overline{n}}J[e^{-\frac{1}{2}\gamma'a^{\dagger}a}]
\end{equation}
with $\gamma' = (2\overline{n}+1)$. Using the commutations relations
and substituing into the expression for the $\mathcal{K}^{(n)}$:
\begin{align}
  e^{-\mathcal{L}_{0}\tau}Se^{\mathcal{L}_{0}\tau} =& e^{-\mathcal{L}_{0}\tau}\left(\gamma_{1}J[a]+\gamma_{2}J[a^{\dagger}] \right)e^{\mathcal{L}_{0}\tau}\\
 e^{-\mathcal{L}_{0}\tau}Se^{\mathcal{L}_{0}\tau} =& e^{-\tau\gamma'}\gamma_{1}J[a]+e^{\tau\gamma'}\gamma_{2}J[a^{\dagger}],
\end{align}
\begin{equation}
\mathcal{K}_{t}^{(m)}=e^{\mathcal{L}_{0}t}\int_{0}^{t}e^{-t_{m}\gamma'}\gamma_{1}J[a]+e^{t_{m}\gamma'}\gamma_{2}J[a^{\dagger}]dt_{m}\int_{0}^{t_{m}}dt_{m-1}e^{-t_{m-1}\mathcal{L}_{0}}Se^{t_{m-1}\mathcal{L}_{0}}...\int_{0}^{t_{2}}dt_{1}e^{-t_{1}\mathcal{L}_{0}}Se^{t_{1}\mathcal{L}_{0}}.
\end{equation}
Particularly, in the zero temperature limit ($\overline{n}=0$) it simplifies to:
\begin{align}
\mathcal{K}_{t}^{(m)}=&J[e^{-\frac{1}{2}\gamma t a^{\dagger}a}]\int_{0}^{t}e^{-t_{m}\gamma'}\gamma J[a]dt_{m}\int_{0}^{t_{m}}dt_{m-1}e^{-t_{m}\gamma'}\gamma J[a]...\int_{0}^{t_{2}}dt_{1}e^{-t_{1}\gamma'}\gamma J[a]\\
\mathcal{K}_{t}^{(m)}=& \frac{(1-e^{-\gamma t})^{m}}{m!}J[e^{-\frac{1}{2}\gamma t a^{\dagger}a}]J[a^{m}]
\end{align}
as proposed in \cite{wiseman_quantum_2010,adesso_optimal_2009}. For long times $\gamma t \to \infty$, all the $\mathcal{K}^{m}_{t}$ tend to zero
and hence every initial state tends towards the vacuum, in the long term all the photons in the mode escape from it.
%%% Local Variables:
%%% mode: latex
%%% TeX-master: "../main"
%%% End:
