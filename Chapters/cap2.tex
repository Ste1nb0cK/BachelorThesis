\chapter{Markovian Open Systems}
% As commented in the previous chapter, in many situations we are interested in only a few degrees of freedoms that couple to many more
% e.g. an atom interacting with an electromagnetic field, in such a way that the description of the complete system is untracktable. For this
% reason it is of great importance to develop methods that allow for the construction of equations of motions for reduced density operators, which
% when exact give almost as intractable non-markovian dynamics like those described by the Nakajima-Zwanzig equation \cite{breuer2002theory} and
% that thus require approximations for their analytical or numerical study. In this chapter we review the
\section{Dynamical Semigroups}
The most accessible and studied type of open quantum systems are those which accept a description through a \textbf{markovian} and \textbf{time
  homogeneous} differential equation. Classical stochastic processes of this kind are characterized through the semigroup
formed by the family of conditional probabilities \footnote{Called a propagator in \cite{breuer2002theory}.}, which is parametrized by the
elapsed time, with this motivation one introduces the quantum analog called \textbf{Dynamical Semigroups}. The analogy is nuanced, as the
time evolution of any probability distribution associated with an observable will not satisfy the Kolmogorov consistency condition, and thus
no description as classical stochastic processes is avaliable \cite{strasberg2022quantum}, furthermore, the evolution of the state
will still be deterministic.

\begin{definition}
  A differentiable parametric family of quantum operations $\{\mathcal{E}_{\tau}\}_{\tau=0}^{\infty}$ such that
  \begin{align}
    \mathcal{E}_{\tau}(\mathcal{E}_{\tau'}(\rho)) =& \mathcal{E}_{\tau + \tau'}(\rho)\\
    \mathcal{E}_{0} = id
  \end{align}
  i.e. that has the semigroup property, is called a \textbf{Dynamical Semigroup}. Strictly, one also demands
  some additional technical conditions on the continouity to treat the case of infinite dimensional Hilbert Spaces \cite{lindblad1976generators}.
\end{definition}

In general these are irreversible; mathematically because the image is \textit{contractive} and physically due to its
positive entropy production \cite{breuer2002theory}, for this demanding a full group structure is too strong. The main result
of this chapter is the classification in terms of generators due to Linblad, which is presented in the following theorem.

\begin{theorem}
  Given a dynamical semigroup $\{\mathcal{E}_{\tau}\}_{\tau=0}^{\infty}$ there exists a time independent superoperator $\mathcal{L}$, called
  \textbf{the generator}, such that:

  \begin{equation}\label{eq:linblad}
  \partial_{t}\rho(t) = \mathcal{L}\rho(t) = -i[ H,\rho ] + \sum_{k}\gamma_{k}\left(L_{k}\rho(t) L_{k}^{\dagger} - \frac{1}{2}\{L_{k}^{\dagger}L_{k}, \rho(t)\}\right)
  \end{equation}
  with $L_{k}$ operators  and $H$ a self-adjoint one \cite{breuer2002theory, lindblad1976generators, wiseman_quantum_2010,hornberger2009introduction}, futhermore, all equations of this form define a dynamical semigroup.
  Equation (\ref{eq:linblad}) is called \textbf{the Linblad Equation}, the traceless $L_{k}$'s the \textbf{Linblad operators}, and
  the $\gamma_{k}$ are positive constants.
\end{theorem}

\section{Kraus representation of the Evolution}
It is clear that at the very least (\ref{eq:linblad}) accepts a formal solution in terms of an exponential superator $\exp(t\mathcal{L})$, one
can use this to form a generalized Dyson series expansion that will gives us a Kraus representation of it by defining a sort
of interaction picture. Begin by defining an
arbitrary decomposition of the generator in terms of two terms $\mathcal{L}_{0}, S$:
\begin{equation}
  \mathcal{L} = \mathcal{L}_{0} + S
\end{equation}
and now introduce the auxiliary unnormalized state $\rho'$
\begin{equation}
\rho = e^{\mathcal{L}_{0}t}\rho'
\end{equation}
subsituting it in the first equality of the Linblad equation one obtains:
\begin{align}
  e^{\mathcal{L}_{0}t}\partial_{t}\rho' + \mathcal{L}_{0}\rho =& S\rho + \mathcal{L}_{0}\rho \\
\partial_{t}\rho' =& e^{-\mathcal{L}_{0}t}Se^{\mathcal{L}_{0}t}\rho_{1}
\end{align}
and integrating from 0 to $t$:
\begin{align}
\rho' = \rho(0) + \int_{0}^{t}dt_{1} e^{-\mathcal{L}_{0}t_{1}}Se^{\mathcal{L}_{0}t_{1}}\rho_{1}(t_{1}).
\end{align}
Iterating this equation:

\begin{align}
\rho' =& \rho(0) +  \int_{0}^{t}dt_{1} e^{-\mathcal{L}_{0}t_{1}}Se^{\mathcal{L}_{0}t_{1}}\left(\rho(0) + \int_{0}^{t_{1}}dt_{2} e^{-\mathcal{L}_{0}t_{2}}Se^{\mathcal{L}_{0}t_{2}}\rho'(t_{2}) \right)\\
  \rho' =& \rho(0) +  \int_{0}^{t}dt_{1} e^{-\mathcal{L}_{0}t_{1}}Se^{\mathcal{L}_{0}t_{1}}\rho(0) + \int_{0}^{t}dt_{1} \int_{0}^{t_{1}}dt_{2}e^{-\mathcal{L}_{0}t_{1}}Se^{\mathcal{L}_{0}(t_{1}-t_{2})}Se^{\mathcal{L}_{0}t_{2}}\rho'(t_{2})\\
  \rho' =& \rho(0) + \sum_{n=1}^{\infty}\int_{0}^{t}dt_{n}\int_{0}^{t_{n}}dt_{n-1}...\int_{0}^{t_{2}}dt_{1}e^{-t_{n}\mathcal{L}_{0}}Se^{(t_{n}-t_{n-1})\mathcal{L}_{0}}S...e^{(t_{2}-t_{1})\mathcal{L}_{0}}Se^{t_{1}\mathcal{L}_{0}}\rho(0).
\end{align}
Note that in the last line we inverted the order of the indexation to make it coincide with \cite{hornberger2009introduction}, so finally
we have for the original state:
\begin{equation}\label{eq:unraveling}
  \rho = e^{t\mathcal{L}_{0}}\rho(0) + \sum_{n=1}^{\infty}\int_{0}^{t}dt_{n}\int_{0}^{t_{n}}dt_{n-1}...\int_{0}^{t_{2}}dt_{1}e^{(t-t_{n})\mathcal{L}_{0}}Se^{(t_{n}-t_{n-1})\mathcal{L}_{0}}S...e^{(t_{2}-t_{1})\mathcal{L}_{0}}Se^{t_{1}\mathcal{L}_{0}}\rho(0).
\end{equation}
This has a formally identical structure to that of the Dyson series in a pertubative expansion, suggesting that one can interpret the evolution
of the system as being given by a perturbation $S$ to the evolution $e^{t{L}_{0}}$. The main problem with this is that in general the splitting
superoperators do not necessarily define a dynamical semigroup e.g. they could fail to map into a traceless operator, which is a consistency
condition. To obtain a more suitable interpretation we choose a particular decomposition, coming from the second equality of (\ref{eq:linblad})
\cite{hornberger2009introduction}:

\begin{align}
\mathcal{L}_{0} =& -i(\tilde{H}\rho - \rho\tilde{H}^{\dagger})\\
S =& \sum_{k}\mathcal{L}_{k}\\
  \tilde{H} =& H - \frac{i}{2}\sum_{k}\gamma_{k}L_{k}^{\dagger}L_{k}\\
  \mathcal{L}_{k} = & J[\sqrt{\gamma_{k}}L_{k}]\rho\
\end{align}
where we used the notation $J[A]\rho = A\rho A^{\dagger}$.
For the evaluation we introduce the following result:
\begin{lemma}
  For any superoperator of the form $\mathcal{A}\rho = A\rho+\rho A^{\dagger}$,
  \begin{equation}
    e^{\tau\mathcal{A}}\rho=\exp(\tau A)\rho\exp(\tau A^{\dagger})
  \end{equation}
\end{lemma}
  \begin{proof}
    Taking the derivative one forms the equation
    \begin{equation}
      \partial_{t}e^{\tau\mathcal{A}}\rho =  A\rho+\rho A^{\dagger}
    \end{equation}
    which by inspection one sees has the solution
    $e^{\tau\mathcal{A}}\rho=\exp(\tau A)\rho\exp(\tau A^{\dagger})$.
  \end{proof}
  allowing us to write (\ref{eq:unraveling}) as in \cite{wiseman_quantum_2010}:

  \begin{equation}
    \rho = J[e^{t \frac{1}{i}\tilde{H}}]\rho(0)  +  \sum_{n=1}^{\infty}\mathcal{K}^{(n)}_{t}\rho(0)
      \end{equation}
\begin{align}
  \mathcal{K}^{(n)}_{t}=&\int_{0}^{t}dt_{n}\int_{0}^{t_{n}}dt_{n-1}...\int_{0}^{t_{2}}dt_{1}J[e^{(t-t_{n}) \frac{1}{i}\tilde{H}}]SJ[e^{(t_{n}-t_{n-1}) \frac{1}{i}\tilde{H}}]...J[e^{(t_{2}-t_{1}) \frac{1}{i}\tilde{H}}]SJ[e^{t_{1}\frac{1}{i}\tilde{H}}]\\
  \mathcal{K}^{(n)}_{t}=&\sum_{k_{1}...k_{n}}\int_{0}^{t}dt_{n}\int_{0}^{t_{n}}dt_{n-1}...\int_{0}^{t_{2}}dt_{1}J\left[e^{(t-t_{n})\frac{1}{i}\tilde{H}}\sqrt{\gamma_{k_{n}}}L_{k_{n}}e^{(t_{n}-t_{n-1})\frac{1}{i}\tilde{H}}...e^{(t_{2}-t_{1})\frac{1}{i}\tilde{H}}\sqrt{\gamma_{k_{1}}}L_{k_{1}}e^{t_{1}\frac{1}{i}\tilde{H}}\right]
\end{align}
%%% Local Variables:
%%% mode: latex
%%% TeX-master: "../main"
%%% End:
