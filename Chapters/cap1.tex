\chapter{The Language of Quantum Information Theory}
% Section on density operator
\section{Density Operators}
%%%%%%%%%%Postulates of Quantum Mechanics
\subsection{Postulates of Quantum Mechanics}
We begin by stating the a common axiomatization of Quantum Mechanics (QM) based on Hilbert Spaces following \cite{ballentine_quantum_2014}, we choose it for mathematical simplicity; for alternatives see e.g. \cite{reyes-lega_aspects_2015}.
\begin{enumerate}
        \item To each physical system $\mathcal{S}$ there corresponds a separable Hilbert Space $\mathcal{H}$ such that states of the system are described by positive and unit trace operators on it. The Hilbert Space of a composite system made up of $\mathcal{S}$ and $\mathcal{S}'$
        is given by the tensor product of the Hilbert spaces $\mathcal{H}\otimes\mathcal{H}'$.
        \item To each dynamical variable there corresponds a self-adjoint operator on $\mathcal{H}$, called an observable, whose possible
        values are given by its eigenvalues.
        \item Given a system in  state $\rho$ and some observable $A$ of it, the probability of measuring $A$ and obtaining the result
        $\lambda$ is given by $\mathrm{Tr}[\rho P_{\lambda}]$ where $P_{\lambda}$ is the eigen-projector into the subspace associated with
        $\lambda$. Furthermore the expectation value is $\mathrm{Tr}[\rho A]$.
        \item After a measurement with result $\lambda$ the state of the system becomes $\frac{P_{\lambda}\rho P_{\lambda}}{\mathrm{Tr}[P_{\lambda}\rho
        P_{\lambda}]}$.
        \item The time evolution of the system in a time interval $(0,t)$ in which no measurement is done is given by some unitary operator
        $U_{t}$ according to $\rho_{t}=U_{t}\rho U_{t}^{\dagger}$ where $\rho$ is the state of the system at time $t=0$.
\end{enumerate}
Operators satisfying the properties required for a state are called \textit{\textbf{Density Operators}} and in contrast to frameworks
whose treatment of quantum states is merely as rays in $\mathcal{H}$, they describe statistical mixtures so imperfect state preparation
can be handled. To see this consider the spectral resolution of some density operator:
\begin{equation}
  \rho = \sum_{n}p_{n}\ketbra{\psi_{n}}
\end{equation}
by definition we have $p_{n}\geq 0, \sum_{n}p_{n}=1$  any density operator can be seen as a convex sum of rays in $\mathcal{H}$ (provided we
identify each one with its associated projector $\ketbra{\psi}$) and from it an  interpretation of $\rho$ as an statistical mixture of rays is
suggested: given a preparation process, there is a probability $p_{n}$ for the system to be in the state $\ketbra{\psi_{n}}$ after it, for this
reason states of the form $\ketbra{\psi}$ are called \textbf{\textit{Pure}} while those who are not we refer to as \textbf{\textit{Mixed}}.
This \textbf{\textit{Ensemble Interpretation}} has serious conceptual challenges when one tries to use it outside a
fixed preparation procedure due to the non-uniqueness of the decomposition into pure states \cite{nielsen_quantum_2010},
but is good enough for the porposes of the present work, for a comprehensive discussion of this
topic the reader is refered to \cite{schlosshauer_decoherence_2007}.
%%%%%%%%%%%%%%Time evolution
\subsection{Time Evolution}
Assuming the evolution to be differentiable in time, we have that there exists a self-adjoint operator $H$ such that $U_{t} = \mathrm{exp}(-itH)$
\footnote{Unless otherwise stated, from here on we assume $\hbar=1$}, called the \textbf{\textit{Hamiltonian}} of the systems and which
acts as the generator of the dynamics. It is straightforward now to construct a differential equation for $\rho_{t}$ by taking the derivative
of it:

\begin{align}
  \rho_{t}=& e^{-itH}\rho_{t}  e^{itH}\\
  \partial_{t}\rho_{t} =& -iH\rho_{t} + \rho_{t}iH\\
  \partial_{t}\rho_{t} =& -i[H, \rho_{t}]\label{eq:Liouville-VonNeumann}.
\end{align}
Equation \eqref{eq:Liouville-VonNeumann} is called the Liouville-Von Neumann equation, it generalizes the time-dependent Schr\"{o}dinger equation
to mixed states and can be interpreted as the quantum analog of the Liouville equation in classical mechanics (with the Poisson braket) through
the quantization rule $\{\bullet,\bullet\} \to -i[\bullet,\bullet]$. As will be seen in later chapters, this type of evolution is characteristic
of closed quantum systems.
%%%%%%%%%%Purity
\subsection{Purity}
Say we got a particular state production processes whose product $\rho$ we characterize via say tomography \cite{nielsen_quantum_2010}, it
becomes immedeatley important to quantify to which extent we can regard the product as being composed of only one pure states
(hopefully the one we wanted to prepare) i.e. we want to define the purity of the state, with this motivation one look for a map
$\mathcal{E}$ from the space of density operators to the reals such that:
\begin{itemize}
        \item $\mathcal{E}(\rho)$ is maximal if and only if $\rho$ is pure.
        \item it is conserved under unitary evolution.
\end{itemize}
The first one makes this map a figure of merit one can try to maximize and the second one is imposed to assure that it doesn't changes in a
closed system unless a measurement is made, as allowing the free evolution of the system should not improve the knowledge of the
experimenter about the system. The standard choiche (altough not the only one) is the \textit{\textbf{Purity}},
defined as \cite{nielsen_quantum_2010, ballentine_quantum_2014}:
\begin{definition}
The purity $\gamma$ of a state $\rho$ is:
  \begin{equation}
    \gamma = \Tr{\rho^{2}}.\label{eq:definition_purity}
  \end{equation}
\end{definition}
The requirements are quickly checked:
\begin{align}
  \Tr{\rho^{2}_{t}} =& \Tr{(U_{t}\rho_{0}U_{t}^{\dagger})(U_{t}\rho_{0}U_{t}^{\dagger})} = \Tr{\rho_{0}^{2}}\\
  \Tr{\rho^{2}} =& \sum_{n}p_{n}^{2} \leq 1
\end{align}
in the second line the inequality is saturated if and only if $\rho=\ketbra{\psi}$.
\section{Entanglement}
One of the key differences between the structure of the state space of classical and quantum systems is the existance
of non-separable states when considering multipartite systems \cite{reyes-lega_aspects_2015,diosi_short_2011,nielsen_quantum_2010} which allows
the latter to have new a new type of correlations. Here we define entanglement for mixed states following \cite{diosi_short_2011}:

\begin{definition}
  Given a state $\rho$ in a system composed of two subsystems $A$ and $B$ with total Hilbert space $\mathcal{H}_{A}\otimes \mathcal{H}_{B}\}$, we say it is
  an \textbf{\textit{entangled}} or \textit{\textbf{non-separable}} if there doesn't exists a set states $\{\rho_{j}\otimes \sigma_{k}\}_{jk}$ and of coefficients
  $\{p_{jk}\}_{jk},  \sum_{jk}p_{jk}=1, \hspace{0.1cm} p_{jk}\geq 0$ such that:
  \begin{equation}
    \rho = \sum_{jk}p_{jk}\rho_{j}\otimes \sigma_{k}
  \end{equation}
  if it does exists, the state is called \textbf{\textit{separable}}.
\end{definition}
For the case of pure state this definition coincides with the usually given one \cite{nielsen_quantum_2010}: say $\rho=\ketbra{\psi}$ is pure and separable,
then:
\begin{equation}
  \Tr{\rho^{2}} = \sum_{jklm}p_{jk}p_{lm}\Tr{\rho_{j}\rho_{l}}\Tr{\sigma_{k}\sigma_{m}}
\end{equation}
and by the Cauchy-Schwartz inequality with the Frobenious inner product
\begin{equation}
  \Tr{\rho^{2}} \leq \sum_{jklm}p_{jk}p_{lm}\Tr{\rho_{j}^{2}}\Tr{\rho_{l}^{2}}\Tr{\sigma_{k}^{2}}\Tr{\sigma_{m}^{2}} \leq 1
\end{equation}
the first inequality from right to left saturates if and only if all the $\rho_{j}$ and $\sigma_{k}$ are pure, and the first one if
and only if all the $\rho_{j}$ and $\sigma_{k}$ are equal between themselves i.e. $p_{jl}=\delta_{0l}\delta_{0j}$. Hence there are pure states
in $\mathcal{H}_{A}$ and $\mathcal{H}_{B}$ such that:
\begin{equation}
  \ketbra{\psi} = \ketbra{\alpha}\otimes\ketbra{\beta}.
\end{equation}
For a classical system all states are separable thanks to the representation via $\delta$ functions
of probability densities \cite{diosi_short_2011} so it is a purely quantum phenomena, although it is remarked that there are non-classical
correlations that do not require non-separability e.g. discord \cite{adesso2016introduction}.

%%% Local Variables:
%%% mode: latex
%%% TeX-master: "../main"
%%% End:
