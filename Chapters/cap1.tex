\chapter{The Language of Quantum Information Theory}
% \bibliography{~/Documents/Research/ThesisCode/Bibliography.bib}
\section{Density Operators}
We begin by stating the a common axiomatization of Quantum Mechanics (QM) based on Hilbert Spaces following \cite{ballentine_quantum_2014}, we choose it for mathematical simplicity; for alternatives see e.g. \cite{reyes-lega_aspects_2015}.
\begin{enumerate}
        \item To each physical system $\mathcal{S}$ there corresponds a Hilbert Space $\mathcal{H}$ and states of the system are described by positive and unit trace operators. The Hilbert Space of a composite system made up of $\mathcal{S}$ and $\mathcal{S}'$
        is given by the tensor product of the Hilbert spaces i.e. $\mathcal{H}\otimes\mathcal{H}'$.
        \item To each dynamical variable there corresponds a self-adjoint operator on $\mathcal{H}$, called an observable, whose possible
        values are given by its eigenvalues.
        \item Given a system in a state $\rho$ and some observable $A$ of it, the probability of measuring $A$ and obtaining the result
        $\lambda$ is given by $\mathrm{Tr}[\rho P_{\lambda}]$ where $P_{\lambda}$ is the eigen-projector into the subspace associated with
        $\lambda$. Furthermore the expectation value is $\mathrm{Tr}[\rho A]$.
        \item After a measurement with result $\lambda$ the state of the system becomes $\frac{P_{\lambda}\rho P_{\lambda}}{\mathrm{Tr}[P_{\lambda}\rho
        P_{\lambda}]}$.
        \item The time evolution of the system in a time interval $(0,t)$ in which no measurement is done is given by some unitary operator
        $U_{t}$ according to $\rho_{t}=U_{t}\rho U_{t}^{\dagger}$ where $\rho$ is the state of the system at time $t=0$.
\end{enumerate}
Operators satisfying the properties required for a state are called \textit{\textbf{Density Operators}} and in contrast to frameworks
whose treatment of quantum states is merely as rays in $\mathcal{H}$, they describe statistical mixtures so imperfect state preparation
can be handled. To see this consider the spectral resolution of some density operator:
\begin{equation}
  \rho = \sum_{n}p_{n}\ketbra{\psi_{n}}
\end{equation}
by definition we have $p_{n}\geq 0, \sum_{n}p_{n}=1$  any density operator can be seen as a convex sum of rays in $\mathcal{H}$ (provided we
identify each one with its associated projector $\ketbra{\psi}$) and from it an  interpretation of $\rho$ as an statistical mixture of rays is
suggested: given a preparation process, there is a probability $p_{n}$ for the system to be in the state $\ketbra{\psi_{n}}$ after it, for this
reason states of the form $\ketbra{\psi}$ are called \textbf{\textit{Pure}} while those who are not we refer to as \textbf{\textit{Mixed}}.
This \textbf{\textit{Ensemble Interpretation}} has serious conceptual challenges when one tries to use it outside a
fixed preparation procedure due to the non-uniqueness of the decomposition into pure states \cite{nielsen_quantum_2010},
but is good enough for the porposes of the present work, for a comprehensive discussion of this
topic the reader is refered to \cite{schlosshauer_decoherence_2007}.

%%% Local Variables:
%%% mode: latex
%%% TeX-master: "../main"
%%% End:
