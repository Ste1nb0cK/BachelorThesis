\begin{figure}[h]
  \centering
\begin{tikzpicture}[>=Stealth, thick]
    % Main box
    \draw (0,0) rectangle (6.5, 4.35);
       % Inner boxes and labels
    \node[draw, minimum size=0.5cm] (E0) at (4,3.5) {$\mathcal{E}_1$};
    \node[draw, minimum size=0.5cm] (E1) at (4,2.5) {$\mathcal{E}_2$};
    \node[draw, minimum size=0.5cm] (EN) at (4,0.5) {$\mathcal{E}_N$};
    % Dots for continuation
    \foreach \y in {1.8,1.5,1.2} {
        \draw[fill] (4,\y) circle (0.5pt);
    };
    % Input
    \node[minimum size=1.5cm, font=\Large] (input) at (-1,2.035)  {$\rho$};
    % First Intersection point
    \node (I0) at (1.0,2.035) {};
    % Arrow to the intersection
    \draw[-] (-0.7, 2.035) -- (1.0,2.035);
    %first vertical line
     \draw[-] (1.0,3.5)--(1.0, 0.5);
    %first horitonzal lines
    \draw[-] (1.0,3.5)--(E0) node[midway, above] {$\Tr{\mathcal{E}_{1}(\rho)}$};
    \draw[-] (1.0,2.5)--(E1) node[midway, above] {$\Tr{\mathcal{E}_{2}(\rho)}$};
    \draw[-] (1.0,0.5)--(EN) node[midway, above] {$\Tr{\mathcal{E}_{N}(\rho)}$};
    %Second intersection point
    \node (I1) at (6.0,2.035) {};
    %Second vertical line
    \draw[-] (6.0,0.5)--(6.0,3.5);
    %Horizontal lines
    \draw[-] (E0)--(6.0, 3.5) ;
    \draw[-] (E1)--(6.0, 2.5 ) ;
    \draw[-] (EN)--(6.0, 0.5) ;
    %Output
    \node[minimum size=0.5cm, font=\large] (output) at (8.3, 2.035) {$\mathcal{E}(\rho)$};
    %arrow out of the intersection
    \draw[-] (6.0,2.035)--(output);
    % caligraphic E below the big box
    \node[minimum size=0.5cm, font=\huge] (label) at (3, -0.5) {$\mathcal{E}$};
\end{tikzpicture}
\caption{Illustration of the stochastic map interpretation.}
\label{fig:stochastic_map_diagram}
\end{figure}
