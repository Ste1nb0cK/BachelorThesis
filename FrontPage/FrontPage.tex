%\newpage
%\setcounter{page}{1}
\begin{center}
\begin{figure}
\centering%
\epsfig{file=FrontPage/EscudoUN.eps,scale=1}%
\end{figure}
\thispagestyle{empty} \vspace*{2.0cm} \textbf{\huge
Quantum Metrology in Bosonic Channels}\\[6.0cm]
\Large\textbf{Nicol\'{a}s Andr\'{e}s Ni\~{n}o Salas}\\[6.0cm]
\small Universidad Nacional de Colombia\\
Facultad de Ciencias, Departamento de F\'{\i}sica\\
Bogot\'{a}, Colombia\\
A\~{n}o 2024\\
\end{center}

\newpage{\pagestyle{empty}\cleardoublepage}

\newpage
\begin{center}
\thispagestyle{empty} \vspace*{0cm} \textbf{\huge
Metrolog\'{i}a Cu\'{a}ntica en Canales Bos\'{o}nicos}\\[3.0cm]
\Large\textbf{Nicol\'{a}s Andr\'{e}s Ni\~{n}o Salas}\\[3.0cm]
\small Trabajo de grado presentada(o) como requisito parcial para optar al
t\'{\i}tulo de:\\
\textbf{F\'{\i}sico}\\[2.5cm]
Director:\\
Ph.D. William Fernando Oquendo Pati\~{n}o\\[2.0cm]
L\'{\i}nea de Investigaci\'{o}n:\\
Informaci\'{o}n  Cu\'{a}ntica\\
Grupo de Investigaci\'{o}n:\\
Caos y Complejidad\\[2.5cm]
Universidad Nacional de Colombia\\
Facultad, Departamento de F\'{\i}sica\\
Bogot\'{a}, Colombia\\
A\~{n}o 2024\\
\end{center}

\newpage{\pagestyle{empty}\cleardoublepage}

\newpage
\thispagestyle{empty} \textbf{}\normalsize
% \\\\\\%
% \textbf{(Dedicatoria o un lema)}\\[4.0cm]

\epigraph{
  Posible, pero no interesante respondi\'{o} L{\"o}nnrot. Usted replicar\'{a} que la realidad no tiene la menor obligaci\'{o}n de ser interesante. Yo le replicar\'{e} que la realidad puede prescindir de esa obligaci\'{o}n, pero no las hip\'{o}tesis. [\ldots] He aqu\'{i} un rabino muerto; yo preferir\'{i}a una explicaci\'{o}n puramente rab\'{i}nica, no los imaginarios percances de un imaginario ladr\'{o}n.}{Jorge Luis Borges,\\ \textit{La Muerte y la Br\'{u}jula }}


\newpage{\pagestyle{empty}\cleardoublepage}

% \newpage
\thispagestyle{empty} \vspace*{10ex} \textbf{\centerline{\LARGE
Declaraci\'{o}n}}\normalsize\\\\\\%
Me permito afirmar que he realizado la presente tesis de manera
aut\'{o}noma y con la \'{u}nica ayuda de los medios permitidos y no
diferentes a los mencionados en la propia tesis. Todos los pasajes
que se han tomado de manera textual o figurativa de textos
publicados y no publicados, los he reconocido en el presente
trabajo. Ninguna parte del presente trabajo se ha empleado en ning\'{u}n
otro tipo de tesis.
\\\\%
Bogot\'{a}, D.C., 31 de Mayo del 2024
\\\\%
\\\\%
\\
\rule{6cm}{0.5pt}\\
Nicol\'{a}s Andr\'{e}s Ni\~{n}o Salas

\thispagestyle{empty} \textbf{}\normalsize
\\\\\\%
\chapter*{Agradecimientos}
% \addcontentsline{toc}{chapter}{\numberline{}Agradecimientos}
% \textbf{\LARGE Agradecimientos}
% \addcontentsline{toc}{chapter}{\numberline{}Agradecimientos}\\\\
Por supuesto, deb\'{i}a comenzar con mis padres Zudelmira y Jaime por su apoyo
incondicional; a pesar de que mis aventuras cient\'{i}ficas parecen llevarme cada vez m\'{a}s lejos de
Barranquilla. El tema de esta tesis naci\'{o} por sugerencia del profesor Carlos Viviescas, con qui\'{e}n
siempre estar\'{e} agradecido por todas nuestras enriquecedoras discusiones a lo largo de los \'{u}ltimos dos
a\~{n}os, as\'{i} como su constante f\'{e} y emoci\'{o}n por mi trabajo. Para terminar me gustar\'{i}a
mencionar a
Daniel ``Pater'' Rojas, Ariadna Contreras, Laura Alfonso, David Calder\'{o}n y Violeta Mendoza, por brindarme su
amistad y confidencialidad a lo largo de estos 5 a\~{n}os, sin ellos ya seguramente ser\'{i}a otro
egoc\'{e}ntrico amargado del \textit{FEM}.
\\

\newpage{\pagestyle{empty}\cleardoublepage}

% \newpage
% \textbf{\LARGE Resumen}
% \addcontentsline{toc}{chapter}{\numberline{}Resumen}\\\\
% El resumen es una presentaci\'{o}n abreviada y precisa (la NTC 1486 de 2008 recomienda revisar la norma ISO 214 de 1976). Se debe usar una extensi\'{o}n m\'{a}xima de 12 renglones. Se recomienda que este resumen sea anal\'{\i}tico, es decir, que sea completo, con informaci\'{o}n cuantitativa y cualitativa, generalmente incluyendo los siguientes aspectos: objetivos, dise\~{n}o, lugar y circunstancias, pacientes (u objetivo del estudio), intervenci\'{o}n, mediciones y principales resultados, y conclusiones. Al final del resumen se deben usar palabras claves tomadas del texto (m\'{\i}nimo 3 y m\'{a}ximo 7 palabras), las cuales permiten la recuperaci\'{o}n de la informaci\'{o}n.\\

% \textbf{\small Keywords: (m\'{a}ximo 10 palabras, preferiblemente seleccionadas de las listas internacionales que permitan el indizado cruzado)}.\\

% A continuaci\'{o}n se presentan algunos ejemplos de tesauros que se pueden consultar para asignar las palabras clave, seg\'{u}n el \'{a}rea tem\'{a}tica:\\

% \textbf{Artes}: AAT: Art y Architecture Thesaurus.

% \textbf{Ciencias agropecuarias}: 1) Agrovoc: Multilingual Agricultural Thesaurus - F.A.O. y 2)GEMET: General Multilingual Environmental Thesaurus.

% \textbf{Ciencias sociales y humanas}: 1) Tesauro de la UNESCO y 2) Population Multilingual Thesaurus.

% \textbf{Ciencia y tecnolog\'{\i}a}: 1) Astronomy Thesaurus Index. 2) Life Sciences Thesaurus, 3) Subject Vocabulary, Chemical Abstracts Service y 4) InterWATER: Tesauro de IRC - Centro Internacional de Agua Potable y Saneamiento.

% \textbf{Tecnolog\'{\i}as y ciencias m\'{e}dicas}: 1) MeSH: Medical Subject Headings (National Library of Medicine's USA) y 2) DECS: Descriptores en ciencias de la Salud (Biblioteca Regional de Medicina BIREME-OPS).

% \textbf{Multidisciplinarias}: 1) LEMB - Listas de Encabezamientos de Materia y 2) LCSH- Library of Congress Subject Headings.\\

% Tambi\'{e}n se pueden encontrar listas de temas y palabras claves, consultando las distintas bases de datos disponibles a trav\'{e}s del Portal del Sistema Nacional de Bibliotecas\footnote{ver: www.sinab.unal.edu.co}, en la secci\'{o}n "Recursos bibliogr\'{a}ficos" opci\'{o}n "Bases de datos".\\

% \textbf{\LARGE Abstract}\\\\
% Es el mismo resumen pero traducido al ingl\'{e}s. Se debe usar una extensi\'{o}n m\'{a}xima de 12 renglones. Al final del Abstract se deben traducir las anteriores palabras claves tomadas del texto (m\'{\i}nimo 3 y m\'{a}ximo 7 palabras), llamadas keywords. Es posible incluir el resumen en otro idioma diferente al espa\~{n}ol o al ingl\'{e}s, si se considera como importante dentro del tema tratado en la investigaci\'{o}n, por ejemplo: un trabajo dedicado a problemas ling\"{u}\'{\i}sticos del mandar\'{\i}n seguramente estar\'{\i}a mejor con un resumen en mandar\'{\i}n.\\[2.0cm]
% \textbf{\small Keywords: palabras clave en ingl\'{e}s(m\'{a}ximo 10 palabras, preferiblemente seleccionadas de las listas internacionales que permitan el indizado cruzado)}\\
